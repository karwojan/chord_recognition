\chapter{Podsumowanie} \label{chapter:summary}

W~ramach niniejszej pracy przeprowadzone zostały badania na temat zastosowania nienadzorowanego uczenia sieci neuronowych w~zadaniu rozpoznawania akordów muzycznych. Przeanalizowana została odpowiednia literatura i~wyszukane zostały dostępne zbiory danych oznaczonych. Następnie przygotowana została metoda automatycznego gromadzenia nagrań z~ogólnodostępnych źródeł. Po zgromadzeniu danych zaprojektowana została adaptacja nienadzorowanego algorytmu uczenia sieci neuronowej na nagraniach muzycznych. Zaimplementowane zostały niezbędne skrypty, pozwalające przeprowadzać dwa rodzaje treningów sieci o~architekturze transformera. Wykonane zostały wreszcie serie eksperymentów, weryfikujących skuteczność zaproponowanego podejścia.

% lipne dane
Pierwszy wniosek z~przeprowadzonych badań dotyczy zastosowanego, automatycznego sposobu gromadzenia danych. Zasadniczo metoda ta nie sprawdziła się zbyt dobrze i~doprowadziła do niższej niż w~literaturze jakości klasyfikacji akordów. Zgromadzone nagrania nie są wystarczająco dobrze dopasowane do znalezionych oznaczeń i~uniemożliwiają skuteczny trening modelu oraz jego rzetelną ewaluację. Właściwie to nierzetelna ewaluacja jest głównym problemem, ponieważ znane są przykłady prac, kiedy model jest trenowany na zaszumionych danych. Aby jednak móc ocenić jego jakość i~porównać z~innymi wynikami w~literaturze, konieczny jest chociaż stosunkowo niewielki, niezaszumiony zbiór. Zbiór taki nie został jednak przygotowany w~ramach niniejszej pracy, co uniemożliwia dobre porównanie z~badaniami innych autorów. Właściwym tematem pracy są jednak metody nienadzorowane i~korzyści z~nich pochodzące. Zaszumiony zbiór nie uniemożliwia dokładnego zbadania tego tematu.

% udane treningi nienadzorowane
Jeżeli chodzi o~przeprowadzone treningi nienadzorowane, to zakończyły się one sukcesem. Udało się wykazać, że model skutecznie uczył się uzupełniać zamaskowane fragmenty spektrogramów. Pozostaje kwestią sporną, czy wykonanie tego zadania wymagało ekstrakcji bardzo abstrakcyjnych i~wysokopoziomowych cech. Faktem jest, że dotrenowanie pretrenowanych w~sposób samonadzorowany enkoderów na zadaniu rozpoznawania akordów prowadziło do nieznacznej poprawy wyników i~zdecydowanie krótszego czasu treningu.

% zysk czasowy
Osiągnięty zysk w~czasie treningu należy uznać za sukces. Kilkukrotne skrócenie czasu treningu przekłada się na realnie mniejsze koszty wytwarzania modeli, które mogą być używane w~różnego rodzaju aplikacjach. Możliwe jest również dzięki temu wykonywanie większej liczby treningów w~krótszym czasie i~w związku z~tym, przeprowadzanie dokładniejszych, rzetelniejszych badań. Co więcej, pretrenowany model ma charakter ogólny i~tak naprawdę powinien przynieść podobną korzyść, kiedy byłby dotrenowywany na innych zadaniach niż rozpoznawanie akordów, takich jak rozpoznawanie gatunku muzycznego, instrumentów czy ekstrakcja linii melodycznej.

% zysk dokładności - probem leży gdzie indziej
Niewielki zysk w~dokładności rozpoznawania akordów można interpretować na wiele sposobów. Najprawdopodobniej jest on jednak związany z~tym, że zadanie rozpoznawania akordów jest zadaniem stosunkowo prostym dla modelu sieci neuronowej. Jakość wyników zależy natomiast głównie od danych uczących, które w~tym przypadku były bardzo zaszumione. Chcąc osiągnąć lepsze wyniki na tym zadaniu, należałoby przede wszystkim skupić się na poprawieniu danych. Metody nienadzorowane są być może w~tym przypadku nadmiarowe, choć jak się okazuje i~tak prowadzą do poprawy wyników. Widać to szczególnie kiedy danych uczących jest skrajnie mało, wtedy trening wstępny pozwala znacząco poprawić jakość klasyfikacji.

% perspektywy
Perspektywy dalszych badań są bardzo różne. Jeżeli chcieć skupić się konkretnie na zadaniu rozpoznawania akordów, to wobec przeprowadzonych eksperymentów i~poczynionych obserwacji, słuszniejszym wydaje się zrezygnować z~tak złożonych i~wymagających metod, jak treningi nienadzorowane. Należałoby natomiast szukać szczegółowych przyczyn niemożności osiągnięcia lepszych wyników, zwłaszcza w~kontekście jakości danych uczących i~walidacyjnych. Z~drugiej zaś strony, w~niniejszej pracy wykazano korzyści i~duży potencjał metod nienadzorowanych w~obszarze MIR. Aby kontynuować badania na ten temat, należałoby przede wszystkim wykonać znacznie więcej eksperymentów nienadzorowanych i~wykorzystać pretrenowane modele w~innych zadaniach, nie tylko w~rozpoznawaniu akordów. Jeżeli wykorzystać przy tym nawet niewielkie, ale jakościowo dobre zbiory danych, możliwe byłoby znalezienie optymalnego sposobu przeprowadzania treningu samonadzorowanego. Trzeba również pamiętać, że istnieją jeszcze inne algorytmy, które są możliwe do wykorzystania i~mogą dawać lepsze rezultaty niż zastosowane podejście, oparte o~metodę MAE.
