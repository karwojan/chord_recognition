\chapter{Zbiór danych}

\section{Wstęp}

% potrzeba danych!
Pierwszym i jednocześnie najważniejszym etapem w tworzeniu rozwiązania opartego o uczenie maszynowe
jest zgromadzenie odpowiedniego zbioru danych. Jest to etap kluczowy, co wydaje się oczywiste po
uwzględnieniu natury tychże algorytmów - wynajdują one pewne reguły w oparciu o podane przykłady.

% potrzeba danych oznaczonych (jak zawsze) i nieoznaczonych (bo pomogą a są)
Głęboki sieci neuronowe, co do zasady potrzebują ,,dużo'' (zazwyczaj wiele milionów) przykładów
uczących, aby móc znaleźć poprawne reguły i wzorce - aby nauczyć się odpowiedniej ekstrakcji cech. W
celu wytrenowania sieci do konkretnego zadania (np. do klasyfikacji obrazów), potrzeba zbioru
zawierającego dane wejściowe do modelu i oczekiwane dane wyjściowe z modelu. Takie uczeni w oparciu
o przykłady oznaczone (ze znaną wartością oczekiwaną) to \emph{uczenie nadzorowane}. Często jednak
ilość danych oznaczonych jest ograniczona, można wtedy wykorzystać dane bez oznaczeń w różnych
algorytmach \emph{uczenia nienadzorowanego}, aby nauczyć sieć struktury tych danych, bez
ukierunkowania na konkretne zadanie. Dopiero później wykorzystuje się mniejszą ilość danych
oznaczonych aby dotrenować sieć (\emph{finetuning}) i nauczyć ją rozwiązywać dany problem.

% co zawiera ten rozdział - od szukania danych do wejścia do sieci + opis skryptów
W ramach niniejszej pracy, ze względu na ograniczoną ilość danych oznaczonych, zastosowane zostało
właśnie powyższe podejście. Do przeprowadzenia eksperymentów potrzebne były dwa zbiory: pierwszy to
,,niewielki'' zbiór danych oznaczonych a drugi to ,,zdecydowanie większy'' zbiór danych bez
oznaczeń. W tym rodzdziale opisany został proces pozyskiwania obu zbiorów danych z ogólnie
dostępnych źródeł. Zawarto tutaj opis wszystkich kroków wykonanych w celu przygotowanie danych do
treningu sieci, poczynając od wyszukiwania oznaczeń utworów muzycznych i przygotowanie parsera dla
plików z tymi oznaczeniami, poprzez automatyczne wyszukiwania i pobieranie odpowiednich plików z
nagraniami muzycznymi, aż po wstępne przetwarzanie danych do formatu odpowiedniego dla sieci
neuronowej. Dodatkowo rozdział ten, poza opisem zastosowanych rozwiązań, zawiera również ogólny opis
implementacji tychże rozwiązań, którą to implementację stanowią skrypty języka Python.


\section{Dane oznaczone}

% trudno o dane oznaczone w ACR
Ze względu na powszechną dostępność dużej ilości danych w Internecie, zgromadzenie zbioru danych
nieoznaczonych nie stanowiło szczególnego problemu. Znacznie trudniejsze okazało się zgromadzenie
zbioru danych oznaczonych. Zadanie rozpoznawania akordów muzycznych, mimo iż będące jednym z
głównych zadań z dziedziny \emph{MIR}, jest oczywiście bardzo mało popularne w stosunku do takich
zadań jak klasyfikacja, czy segmentacja obrazów. Jest to również po prostu znacznie mniej przydatne
w życiu codziennym i znacznie mniej się w tę dziedzinę inwestuje. Należy również uwzględnić fakt, że
przygotowanie oznaczeń akordów do utworów muzycznych wymaga specjalistycznej wiedzy muzycznej,
praktyki muzycznej i dużo czasu. Oznaczenia powstałe w ten sposób nadal pozostają subiektywne i będą
się często różnić, w zależności od osoby, która utwory oznaczała. Wszystkie wymienione powody są
przyczyną tego, że nie przygotowano więcej niż kilka publicznie dostępnych zbiorów oznaczeń akordów
muzycznych, które mogą zostać wykorzystane do treningu modeli uczenia maszynowego.


\section{Publiczne zbiory z oznaczeniami akordów}

% zbiory są takie, co to inni wykorzystywali
W ramach niniejszej pracy przestudiowana została literatura z dziedziny rozpoznawania akordów za
pomocą sieci neuronowych - większość wykorzystywanych przez badaczy zbiorów danych udało się
odnaleźć i wykorzystać. Zostały one opisane poniżej. Trzeba jeszcze zaznaczyć, że wszystkie te
zbiory są udostępniane za darmo, ale jedynie jako same pliki tekstowe z oznaczeniami akordów. Pliki
z nagranimi muzycznymi, ze względu na prawa autorskie, nie są udostępniane. Pozyskanie odpowiednich
nagrań stanowi osobny problem, opisany w dalszej części rozdziału.

\subsection{Zbiór Isophonics}

\emph{Isophonics} to właściwie nazwa serwisu internetowego\footnote{www.isophonics.net/about},
prowadzonego przez zespół naukowy \emph{Centre for Digital Music} z londyńskiego Uniwersytety
Królowej Marii. Jest to duża i popularna na cały świat jednostka naukowa specjalizująca się w
badaniach dotyczących przetwarzania muzyki cyfrowej. W serwisie tym dostępne są między innymi
oprogramowanie oraz zbiory danych związane z różnymi aspektami przetwarzania sygnałów muzycznych.
Można tam znaleźć najbardziej popularny i zapewne najstarszy zbiór z oznaczeniami akordów
muzycznych, przygotowany dla 180 piosenek zespołu \emph{The Beatles}, szczegółowo opisany w
\cite{harte_towards_nodate}. Zbiór ten rozrósł się o oznaczenia dla 20 utworów \emph{Queen}, 18
utowrów \emph{Zweieck} i 7 utworów \emph{Carole King}. Wykorzystany format zapisu oznaczeń akordów
również jest opisany w \cite{harte_towards_nodate} i jest wykorzystywany praktycznie jako standard,
również w przypadku innych zbiorów danych. Warto wspomnieć jeszcze, że jest to pierwszy referencyjny
zbiór wykorzystywany w konkursie MIREX, w zadaniu automatycznego rozpoznawania akordów.

\subsection{Zbiór McGill Billboard}

Zbiór \emph{McGill Billboard}\cite{burgoyne_expert_2011} został stworzony przez grupę \emph{DDMAL
(Distributed Digital Music Archives \& Libraries Lab)}\footnote{https://ddmal.music.mcgill.ca/} z
kandadyjskiego \emph{McGill University}. Jak nazwa wskazuje, zajmują się oni różnymi projektami
związanymi z preztwarzaniem muzyki, w tym tworzą i utrzymują ten prawdopodobnie największy dostępny
zbiór danych z oznaczeniami akordów.

% pamiętać, żeby opisać wszystkie kroki, SKRYPTY i zerknąć na COMMITY


\section{Pozyskanie plików z nagraniami muzycznymi}



% - zorganizować zbiór danych nieoznaczonych
% 
% - wstępne przetwarzanie danych - format danych wejściowy i wyjściowy (wejściowy do sieci)
%     - format i parsowanie plików z oznaczeniami i plików muzycznych
%     - podział na ramki i dopasowanie labelek do ramek
%     - augmentacja
