\chapter{Zbiór danych}

\section{Wstęp}

% potrzeba danych!
Pierwszym i jednocześnie najważniejszym etapem w tworzeniu rozwiązania opartego o uczenie maszynowe
jest zgromadzenie odpowiedniego zbioru danych. Jest to etap kluczowy, co wydaje się oczywiste po
uwzględnieniu natury tychże algorytmów - wynajdują one pewne reguły w oparciu o podane przykłady.

% potrzeba danych oznaczonych (jak zawsze) i nieoznaczonych (bo pomogą a są)
Głęboki sieci neuronowe, co do zasady potrzebują ,,dużo'' (zazwyczaj wiele milionów) przykładów
uczących, aby móc znaleźć poprawne reguły i wzorce - aby nauczyć się odpowiedniej ekstrakcji cech. W
celu wytrenowania sieci do konkretnego zadania (np. do klasyfikacji obrazów), potrzeba zbioru
zawierającego dane wejściowe do modelu i oczekiwane dane wyjściowe z modelu. Takie uczeni w oparciu
o przykłady oznaczone (ze znaną wartością oczekiwaną) to \emph{uczenie nadzorowane}. Często jednak
ilość danych oznaczonych jest ograniczona, można wtedy wykorzystać dane bez oznaczeń w różnych
algorytmach \emph{uczenia nienadzorowanego}, aby nauczyć sieć struktury tych danych, bez
ukierunkowania na konkretne zadanie. Dopiero później wykorzystuje się mniejszą ilość danych
oznaczonych aby dotrenować sieć (\emph{finetuning}) i nauczyć ją rozwiązywać dany problem.

% co zawiera ten rozdział - od szukania danych do wejścia do sieci + opis skryptów
W ramach niniejszej pracy, ze względu na ograniczoną ilość danych oznaczonych, zastosowane zostało
właśnie powyższe podejście. Do przeprowadzenia eksperymentów potrzebne były dwa zbiory: pierwszy to
,,niewielki'' zbiór danych oznaczonych a drugi to ,,zdecydowanie większy'' zbiór danych bez
oznaczeń. W tym rodzdziale opisany został proces pozyskiwania obu zbiorów danych z ogólnie
dostępnych źródeł. Zawarto tutaj opis wszystkich kroków wykonanych w celu przygotowanie danych do
treningu sieci, poczynając od wyszukiwania oznaczeń utworów muzycznych i przygotowanie parsera dla
plików z tymi oznaczeniami, poprzez automatyczne wyszukiwania i pobieranie odpowiednich plików z
nagraniami muzycznymi, aż po wstępne przetwarzanie danych do formatu odpowiedniego dla sieci
neuronowej. Dodatkowo rozdział ten, poza opisem zastosowanych rozwiązań, zawiera również ogólny opis
implementacji tychże rozwiązań, którą to implementację stanowią skrypty języka Python.


\section{Dane oznaczone}

% trudno o dane oznaczone w ACR
Ze względu na powszechną dostępność dużej ilości danych w Internecie, zgromadzenie zbioru danych
nieoznaczonych nie stanowiło szczególnego problemu. Znacznie trudniejsze okazało się zgromadzenie
zbioru danych oznaczonych. Zadanie rozpoznawania akordów muzycznych, mimo iż będące jednym z
głównych zadań z dziedziny \emph{MIR}, jest oczywiście bardzo mało popularne w stosunku do takich
zadań jak klasyfikacja, czy segmentacja obrazów. Jest to również po prostu znacznie mniej przydatne
w życiu codziennym i znacznie mniej się w tę dziedzinę inwestuje. Należy również uwzględnić fakt, że
przygotowanie oznaczeń akordów do utworów muzycznych wymaga specjalistycznej wiedzy muzycznej,
praktyki muzycznej i dużo czasu. Oznaczenia powstałe w ten sposób nadal pozostają subiektywne i będą
się często różnić, w zależności od osoby, która utwory oznaczała. Wszystkie wymienione powody są
przyczyną tego, że nie przygotowano więcej niż kilka publicznie dostępnych zbiorów oznaczeń akordów
muzycznych, które mogą zostać wykorzystane do treningu modeli uczenia maszynowego.


% do początku do uzyskania pierwszego indeksu (skrypt 01)
\subsection{Publiczne zbiory z oznaczeniami akordów}

% zbiory są takie, co to inni wykorzystywali
W ramach niniejszej pracy przestudiowana została literatura z dziedziny rozpoznawania akordów za
pomocą sieci neuronowych - większość wykorzystywanych przez badaczy zbiorów danych udało się
odnaleźć i wykorzystać. Zostały one szczegółowo opisane poniżej. Każdy zbiór oznaczeń został
pobrany, odpowiednio oczyszczony (wybrane zostały tylko niezbędne pliki) i zapisany w repozytorium
projektu. Po przeanalizowanu struktury zbiorów stworzony został pierwszy skrypt przygotowujący dane
- ,,src/dataset\_scripts/01-generate\_index\_of\_chordlab.py'' - który, w postaci pliku \emph{csv},
generuje jeden wspólny indeks dla wszystkich zbiorów, tak że mogą one być już traktowane jako jeden
zbiór danych. Fragment tego indeksu został przedstawionyw w \ref{tab:indeks_01}.

\begin{table}
    \caption{Fragment indeksu zbioru danych po pierwszym etapie jego tworzenia}
    \label{tab:indeks_01}
    \begin{tabular}{rllllr}
    \hline
    & filepath & song & artist & subset & album \\
    \hline
     0 & ./data/chordlab/rs200/1999\_dt.clt                      & 1999                      & Prince                      & rs200 & nan \\ 
     1 & ./data/chordlab/rs200/a\_change\_is\_gonna\_come\_dt.clt    & A Change Is Gonna Come    & Sam Cooke                   & rs200 & nan \\
     5 & ./data/chordlab/rs200/all\_along\_the\_watchtower\_dt.clt  & All Along the Watchtower  & The Jimi Hendrix Experience & rs200 & nan \\
     6 & ./data/chordlab/rs200/all\_apologies\_dt.clt             & All Apologies             & Nirvana                     & rs200 & nan \\
     7 & ./data/chordlab/rs200/all\_i\_have\_to\_do\_is\_dream\_dt.clt & All I Have to Do Is Dream & The Everly Brothers         & rs200 & nan \\
    \hline
    \end{tabular}
\end{table}

Trzeba jeszcze zaznaczyć, że wszystkie te zbiory są udostępniane za darmo, ale jedynie jako same
pliki tekstowe z oznaczeniami akordów. Pliki z nagranimi muzycznymi, ze względu na prawa autorskie,
nie są udostępniane. Pozyskanie odpowiednich nagrań stanowi osobny problem, opisany w dalszej części
rozdziału.

\subsubsection{Zbiór Isophonics}

% ogólny opis, jakie i ile utworów, format zapisu
\emph{Isophonics} to właściwie nazwa serwisu internetowego\footnote{www.isophonics.net/about},
prowadzonego przez zespół naukowy \emph{Centre for Digital Music} z londyńskiego Uniwersytety
Królowej Marii. Jest to duża i popularna na cały świat jednostka naukowa specjalizująca się w
badaniach dotyczących przetwarzania muzyki cyfrowej. W serwisie tym dostępne są między innymi
oprogramowanie oraz zbiory danych związane z różnymi aspektami przetwarzania sygnałów muzycznych.
Można tam znaleźć najbardziej popularny i zapewne najstarszy zbiór z oznaczeniami akordów
muzycznych, przygotowany dla 180 piosenek zespołu \emph{The Beatles}, szczegółowo opisany w
\cite{harte_towards_nodate}. Zbiór ten rozrósł się o oznaczenia dla 20 utworów \emph{Queen}, 18
utowrów \emph{Zweieck} i 7 utworów \emph{Carole King}. Wykorzystany format zapisu oznaczeń akordów
również jest opisany w \cite{harte_towards_nodate} i jest wykorzystywany praktycznie jako standard,
również w przypadku innych zbiorów danych. Warto wspomnieć jeszcze, że jest to pierwszy referencyjny
zbiór wykorzystywany w konkursie MIREX, w zadaniu automatycznego rozpoznawania akordów.

% pobieranie i indeksowanie
Zbiór \emph{Isohponics} jest dostępny do pobrania ze strony projektu w postaci skompresowanych
archiwów \emph{tar} (osobny plik dla każdego z czterech artystów). Każde z tych archiwów ma taką
samą strukturę i zawiera nie tylko oznaczenia akordów, dostępne w różnych formatach, ale również
inne informacje o utworach, jak zmiany tonacji czy segmentacje strukturalne. Spośród wszystkich tych
plików wykorzystane zostały jedynie oznaczenia akordów w postaci plików \emph{lab} - pozostałe pliki
zostały usunięte. Pliki z oznaczeniami akordów uporządkowane są w takiej stukturze katalogów, gdzie
nazwy katalogów na kolejnych poziomach zagłębienia odpowiadają kolejno nazwie artysty, nazwie albumu
(dla The Beatles poprzedzonej dodatkowo liczbą porządkową) i numerowi wraz z nazwą konkretnego
utworu. Zbiór ten nie posiada żadnego dodatkowego indeksu, przy tworzeniu własnego indeksu zostały
więc wykorzystane nazwy katalogów.

% poprawki i zapisanie w projekcie
Jedyne zmiany, jakie zostały wprowadzone w oznaczeniach akordów po pobraniu ich z Internetu, poza
usunięciem niewykorzystywanych plików, dotyczą artystki Carole King. Polegały one na dostosowaniu
kilku skrótów typów akordów, które występowały jedynie tutaj a były niezgodne z przyjętą w
\cite{harte_towards_nodate} konwencją.  Wszystkie oznaczenia zbioru \emph{isohponics}, po
przeprowadzeniu opisanych powyżej operacji, zostały zapisane w repozytorium projektu, w katalogu
,,data/chordlab/isophonics''.


\subsubsection{Zbiór McGill Billboard}

% ogólny opis, jakie i ile utworów, format zapisu
Zbiór \emph{McGill Billboard}\cite{burgoyne_expert_2011} został stworzony przez grupę \emph{DDMAL
(Distributed Digital Music Archives \& Libraries Lab)}\footnote{https://ddmal.music.mcgill.ca/} z
kandadyjskiego \emph{McGill University}. Jak nazwa wskazuje, zajmują się oni różnymi projektami
związanymi z przetwarzaniem muzyki, w tym tworzą i utrzymują ten prawdopodobnie największy, dostępny
publicznie zbiór danych z różnymi typami oznaczeń utworów, takimi jak akordy, struktura, instrumenty
i tempo. Lista utworów wybranych do tego zbioru została zbudowana z 1300 utworów, losowo
próbkowanych z rankingów serwisu \emph{Billboard}\footnote{www.billboard.com}. W praktyce, ponieważ
niektórych utworów nie udało się twórcom znaleźć, zbiór ten zawiera 890 utworów muzyki popularnej,
przy czym zdarzają się takie, które się powtarzają. Oznaczenia akordów są wyrównane w czasie zgodnie
z częstotliwością występowania taktów. Format tych oznaczeń bazuje ściśle na
\cite{harte_towards_nodate} (tak jak \emph{isophonics}), jedyna różnica polega na wprowadzeniu kilku
dodatkowych skrótów typów akordów.  Podobnie jak w przypadku innych zbiorów z oznaczeniami akordów,
twórcy tego zbioru nie mogli udostępnić nagrań muzycznych. Jednakże w tym przypadku udostępnione
zostały dwa zestawy cech dla wszystkich akordów: \emph{non-negative-least-squares chroma vectors}
oraz \emph{Echo Nest features}, które mogą zostać wykorzystane jako wejście do modeli ML, nie były
jednak użyte w ramach niniejszej pracy.

% pobieranie i indeksowanie
Zbiór \emph{McGill Billboard} dostępny jest do
pobrania\footnote{https://ddmal.music.mcgill.ca/research/The\_McGill\_Billboard\_Project\_(Chord\_Analysis\_Dataset)/}
w postaci skompresowanych (na kilka sposobów) archiwów \emph{tar}. Dostępne są pliki ze wspomnianymi
powyżej dwoma zestawami dźwiękowych cech utworów, dostępny jest zestaw wszystkich oznaczeń we
własnym formacie twórców zbioru, oraz zestaw oznaczeń samych akordów, w postaci plików \emph{lab}.
Do niniejszej wykorzystany został jedynie ten ostatni zestaw, zawierający pliki \emph{lab}. Pobrane
archiwum zawiera główny katalog ,,McGill-Billboard'', w którym znajduje się 890 katalogów nazwanych
liczbowymi identyfikatorami utworów ze zbioru. Każdy taki katalog zawiera jeden plik ,,full.lab'' z
oznaczeniami akordów. Twórcy zbioru dostarczają również indeks w postaci pliku \emph{csv}, wiążący
identyfikatory utworów z tytułem, artystą (album nie jest podany) i innymi informacjami związanymi z
rankingiem, z którego pochodzi utwór. Indeks ten jest niestety bardzo wybrakowany i dla wielu
utworów brakuje wpisu o tytule lub artyście. Po wybraniu tych przypadków, dla których znany jest
tytuł i artysta zostało jedynie 596 utworów. Tabela ta została wykorzystana do przygotowanie
własnego indeksu.

% poprawki i zapisanie w projekcie
Pobrane oznaczenia akordów nie wymagały właściwie żadnych poprawek. Cały katalog
,,McGill-Billboard'' został przeniesiony do repozytorium projetu, do katalogu
,,data/chordlab/mcgill\_billboard''.

\subsubsection{Zbiór Robbie Williams}

Zbiór \emph{Robbie Williams}\cite{giorgi_automatic_2013} składa się z oznaczeń akordów oraz tonacji
dla pierwszych pięciu albumów Robbiego Williamsa. Został przygotowany TODO

\subsubsection{Zbiór RS200}

% ogólny opis, jakie i ile utworów, format zapisu
Zbiór \emph{RS200}\cite{de_clercq_corpus_2011} został stworzony przez dwóch badaczy z amerykańskiej
uczelni muzycznej \emph{Eastman School of Music} w Rochester. Powstał on na bazie listy \emph{500
Greatest Songs of All Time} z magazynu Rolling Stone. Początkowo zawierał jedynie 100 utworów i w
takiej formie zostały wykorzystany w oryginalnej publikacji, gdzie autorzy analizowali częstość
występowania różnych akordów i ich progresji. Utwory te zostały wybrane tak, aby pochodziły z wielu
różnych dekad. Z czasem zbiór został rozszerzony do 200 utworów i w takiej formie jest dostępny do
dzisiaj. Każdy z dwóch twórców, niezależnie od drugiego, przygotował swoje oznaczenia dla wszystkich
200 utworów. Oznaczenia te zawierają nie tylko akordy, ale również transkrypcję melodii, informacje
o tempie i tekstach utworów. Format zapisu oznaczeń akordów został opracowanych przez twórców tego
zbioru danych i różni się zdecydowanie od sposobu oznaczeń stosowanego w pozostałych zbiorach.



% pamiętać, żeby opisać wszystkie kroki, SKRYPTY i zerknąć na COMMITY


\section{Pozyskanie plików z nagraniami muzycznymi}



% - zorganizować zbiór danych nieoznaczonych
% 
% - wstępne przetwarzanie danych - format danych wejściowy i wyjściowy (wejściowy do sieci)
%     - format i parsowanie plików z oznaczeniami i plików muzycznych
%     - podział na ramki i dopasowanie labelek do ramek
%     - augmentacja
