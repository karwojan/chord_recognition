\chapter{Wstęp} \label{chapter:introduction}



\section{Problematyka i zakres pracy}

% zakres
Niniejsza praca dotyczy zakresu cyfrowego przetwarzania sygnałów, sztucznej inteligencji i uczenia maszynowego, a w szczególności nadzorowanych i nienadzorowanych sposobów uczenia głębokich sieci neuronowych.

% główny cel i problematyka
Głównym celem pracy jest opracowanie i przeanalizowanie algorytmu rozpoznawania akordów muzycznych bazującego na nienadzorowanych metodach uczenia sieci neuronowych. Na problematykę tę składają się więc dwa, dość luźno powiązane tematy:
\begin{itemize}
    \item automatyczne rozpoznawania akordów muzycznych;
    \item nienadzorowane uczenie sieci neuronowych.
\end{itemize}

% ogólnie o rozpoznawaniu akordów
Problem rozpoznawania akordów jest jednym z podstawowych zadań z dziedziny MIR (ang. \emph{Music Information Retrieval}) i polega na automatycznym wyodrębnieniu z nagrania muzycznego podstawowych struktur, jakimi są akordy --- różne układy brzmiących jednocześnie dźwięków. Automatyczne rozpoznawanie akordów może mieć wiele specjalistycznych zastosowań, takich jak ocenianie podobieństwa między utworami, rozpoznawanie gatunku czy automatyczne przygotowanie nut dla amatorów grania na gitarze. W stosunku do innych zadań realizowanych za pomocą głębokich sieci neuronowych, takich jak klasyfikacja obrazów czy rozpoznawanie mowy, rozpoznawanie akordów jest stosunkowo proste. Istnieje już wiele dobrze sprawdzających się algorytmów realizujących to zadanie, zarówno starszych, bazujących na klasycznych metodach przetwarzania sygnałów, jak i nowszych, opartych na uczeniu maszynowym, w tym o głębokie sieci neuronowe. 

% minusy rozpoznawania sieciami
Nowsze metody, bazujące głównie na sieciach neuronowych, wykazują się większą dokładnością niż klasyczne, deterministyczne algorytmy. Wymagają jednak uprzedniego przygotowania przez człowieka jak największej ilości poprawnych oznaczeń akordów dla wielu różnych utworów. Wymaga to dużo czasu i specjalistycznej wiedzy oraz praktyki muzycznej. Ponadto oznaczenia takie pozostają mocno subiektywne i trudno, aby były pozbawione błędów i niedokładności.  Na podstawie gotowych przykładów sieć neuronowa uczy się sama realizować to zadanie. Dodatkowo proces nauki jest czasochłonny i wymaga specjalistycznego sprzętu, jak karty graficzne, aby móc być zrealizowanym w czasie nie dłuższym niż kilkanaście godzin. O ile więc podejście to daje lepsze wyniki, jest znacznie trudniejsze w realizacji i narażone na wiele różnych błędów.

% wchodzi self-supervised
Częściowym rozwiązaniem powyższych problemów jest wykorzystanie nienadzorowanych metod uczenia sieci neuronowych. Metody te rozwijają się prężnie w ostatnich latach i pozwalają wykorzystać dane bez oznaczeń (np. losowe nagrania pobrane z serwisów internetowych), aby nauczyć sieć neuronową struktury tychże danych i występujących w nich zależności. Przygotowane w ten sposób modele sieci neuronowych mogą być później uczone realizacji konkretnych zadań metodami nadzorowanymi. Podejście to prowadzi do potencjalnych korzyści, takich jak:
\begin{itemize}
    \item krótszy czas nauki zadania docelowego (jeden model wytrenowany w sposób nienadzorowany może być wykorzystany do wielu zadań docelowych);
    \item mniejsza ilość danych, wymaganych do nauki;
    \item lepsze generalizacja i ogólnie lepsze wyniki.
\end{itemize}
Zadanie rozpoznawania akordów nadaje się więc idealnie, aby zastosować w nim powyższe podejście. Pojawiły się już pierwsze próby stosowania metod nienadzorowanych w zadaniu rozpoznawania akordów. Badania te są jednak jeszcze bardzo nieliczne i niedojrzałe. Zdecydowanie pełen potencjał metod nienadzorowanych, który można obserwować w popularniejszych dziedzinach jak przetwarzanie obrazu i tekstu naturalnego, nie został jeszcze wykorzystany ani tutaj, ani w całej dziedzinie MIR.

% wyzwania
Podejmując tematykę rozpoznawania akordów z wykorzystaniem nienadzorowanych metod uczenia sieci neuronowych, trzeba zmierzyć się z całym mnóstwem wyzwań. Po pierwsze i najważniejsze, należy zgromadzić dwa zbiory danych. Pierwszym jest mniejszy zbiór danych oznaczonych, niezbędny, aby ostatecznie nauczyć sieć rozpoznawać akordy. Drugim jest najlepiej wielokrotnie większy zbiór danych bez oznaczeń, który będzie wykorzystany do treningów nienadzorowanych. Podczas gromadzenia danych trzeba zmierzyć się z problemami ich jakości, zaszumienia błędnymi przykładami, odpowiednich rozkładów i metod przechowywania. Po drugie, trzeba przygotować algorytm uczenia nienadzorowanego, który będzie adaptował istniejące już rozwiązania z innych, większych i lepiej finansowanych dziedzin, w szczególności przetwarzania obrazów, rozpoznawania mowy i przetwarzania języka naturalnego. Na końcu trzeba przeprowadzić serię czasochłonnych i zasobożernych eksperymentów, których celem jest wykazać korzyści z treningu nienadzorowanego. Korzyści te wcale nie muszą się wyraźnie pojawić i należy zadbać o ostrożną i rozsądną interpretację wyników.

% wyniki
W wyniku doświadczeń przeprowadzonych w ramach niniejszej pracy, udało się wykazać korzyści z treningu nienadzorowanego. Głównym znalezionym zyskiem jest wielokrotnie krótszy czas treningu i nieznacznie poprawione wyniki. Przeprowadzone badania są jednak bardzo ograniczone i absolutnie nie wyczerpują podjętego tematu pracy.



\section{Cele pracy}

\subsubsection{Zapoznanie się ze współczesnymi metodami nienadzorowanego uczenia sieci i ich popularyzacja}

Pierwszym celem poznawczym pracy było zebranie informacji na temat metod uczenia sieci neuronowych bez nadzoru, czyli wykorzystując ogólnodostępne dane bez oznaczeń. Metody te rozwijają się w ostatnich latach i są wykorzystywane zwłaszcza w dziedzinach takich jak rozpoznawanie obrazu, rozpoznawanie mowy i analiza języka naturalnego. Stanowią ciekawy i rozsądny kierunek rozwoju głębokich sieci neuronowych, umożliwiający osiąganie zupełnie nowych rezultatów.

\subsubsection{Zapoznanie się z metodami rozpoznawania akordów muzycznych, bazującymi na sieciach neuronowych}

Drugim celem poznawczym było zebranie i analiza literatury dotyczącej rozpoznawania akordów za pomocą sieci neuronowych. Przegląd ten miał służyć zorientowaniu się, na jakim etapie są badania nad tym konkretnym, jak i również innymi problemami z dziedziny MIR. Na tej podstawie można zaproponować wiele potencjalnych usprawnień i kierunków rozwoju zaadoptowanych z bardziej rozwiniętych obszarów jak przetwarzanie języka naturalnego.

\subsubsection{Zgromadzenie jak największych zbiorów danych oznaczonych i nieoznaczonych oraz automatyzacja tego procesu}

Pierwszym praktycznym zadaniem realizowanym w ramach niniejszej pracy jest zgromadzenie odpowiednio dużych zbiorów danych oznaczonych i nieoznaczonych. Gromadzenie, organizacja i wstępne przetwarzanie danych stanowi zawsze kluczowy element w zadaniach rozwiązywanych za pomocą uczenia maszynowego. W niniejszej pracy podjęta została próba jak największej automatyzacji tego czasochłonnego procesu, ocenione również zostały skutki zaproponowanego podejścia.

\subsubsection{Zaprojektowanie i wykorzystanie algorytmu nienadzorowanego uczenia sieci neuronowych w zadaniu rozpoznawania akordów muzycznych}

W ramach niniejszej pracy powstała propozycja algorytmu uczenia nienadzorowanego, który może być wykorzystany i przynieść korzyści w realizacji zadania rozpoznawania akordów muzycznych. Algorytm ten jest bardzo prosty i stanowi jedynie adaptację pewnego podejścia, wykorzystywanego w obszarach przetwarzaniu języka naturalnego, analizy obrazu i rozpoznawania mowy.

\subsubsection{Zbadanie korzyści z zaproponowanego rozwiązanie poprzez przeprowadzenie odpowiednich eksperymentów}

Zaproponowane rozwiązanie zostało w miarę możliwości zbadane empirycznie, poprzez przeprowadzenie odpowiednich eksperymentów. Wykonane zostały więc serie różnych treningów różnych modeli sieci neuronowych, mających na celu ocenić korzyści z zaproponowanego podejścia. Wszystkie eksperymenty zostały dokładnie udokumentowane i szczegółowo opisane.



\section{Metoda badawcza}

W pierwszej kolejności przeanalizowana została cała znaleziona literatura, dotycząca rozpoznawania akordów za pomocą sieci neuronowych. Są to głównie pochodzące z wielu różnych źródeł i zagregowane na platformach internetowych artykuły naukowe, dostępne w języku angielskim. Taką samą formę mają również prace stanowiące drugą część przeglądu, czyli odnoszące się do nienadzorowanego uczenia sieci neuronowych. Ze względu na ich ogromną ilość, przejrzana i opisana została jedynie mała część, co istotniejszych prac na ten temat. Związane są one z bardzo różnymi obszarami, takimi jak przetwarzanie tekstu naturalnego, analiza obrazu, rozpoznawanie mowy, rozpoznawanie innych dźwięków, a nawet pierwsze podejścia do wykorzystania nienadzorowanego uczenia w rozpoznawaniu akordów.

Drugim etapem pracy była adaptacja istniejących metod uczenia nienadzorowanego do dziedziny przetwarzania utworów muzycznych. Wybrane zostało w szczególności jedno podejście do nienadzorowanego treningu sieci neuronowych i wykorzystane przy projektowaniu ostatecznej formy zaproponowanego w ramach niniejszej pracy algorytmu.

Skuteczność zaproponowanego algorytmu została oceniona poprzez odpowiednie eksperymenty, których głównym celem było wykazanie korzyści z zaproponowanego podejścia. Wybrana została więc jedna praca \cite{park_bi-directional_2019}, prezentująca ostatnie, najlepsze wyniki na zadaniu rozpoznawania akordów. Podejście autorów, sprowadzające się do odpowiedniej procedury wstępnego przetwarzania danych i odpowiedniej architektury modelu sieci neuronowej, zostało zreprodukowane i nieznacznie uproszczone. W ten sposób przygotowane zostało środowisko do przeprowadzenia eksperymentów. Następnie, przeprowadzone zostały dwa treningi nienadzorowane i seria treningów nadzorowanych, podczas których wykazana została przydatność i skuteczność treningów nienadzorowanych.



\section{Przegląd literatury w dziedzinie}

Przegląd literatury jak wspomniano wcześniej, składa się z dwóch głównych części. Pierwsza część dotyczy rozpoznawania akordów, ale tylko z wykorzystaniem sieci neuronowych. Aby ograniczyć liczbę analizowanych prac, zrezygnowano z tych, które dotyczą starszych metod, stosowanych przed rozpowszechnieniem się sieci neuronowych (z drobnymi wyjątkami). Skupiono się natomiast możliwie dokładnie na analizie rozwiązań opartych na sieciach neuronowych. Wyszukano w tym celu i zgromadzono niemalże wszystkie (zdaniem autora) opublikowane prace, gdzie próbowano rozpoznawać akordy modelami sieci neuronowych. Druga część przeglądu jest luźniejsza i mniej szczegółowa, ponieważ dotyczy znacznie szerszego obszaru, jakim jest nienadzorowane uczenie sieci neuronowych. Stosowane jest ono w bardzo różnych dziedzinach. Wyszukane zostały co ważniejsze i bardziej interesujące w kontekście rozpoznawania akordów prace, gdzie proponowano, analizowano i adaptowano algorytmy nienadzorowanego uczenia sieci neuronowych. Algorytmy te dzielą się właściwie na dwie, powiązane ze sobą grupy: metody samonadzorowane (ang. \emph{self-supervised}) i metody półnadzorowane (ang. \emph{semi-supervised}). Wszystkie te algorytmy mają natomiast ten sam cel: wykorzystać dane bez oznaczeń, aby później uzyskać lepsze wyniki na konkretnym zadaniu.

\subsection{Rozpoznawanie akordów za pomocą sieci neuronowych}

% początek (humphrey)
Pierwszą próbę rozpoznawania akordów muzycznych za pomocą sieci neuronowych podjęli Eric J. Humphrey i Juan P. Bello w 2012 roku \cite{humphrey_rethinking_2012}. Opisali oni jak za pomocą splotowej sieci neuronowej, stosowanej wcześniej głównie w rozpoznawaniu obrazów, można rozpoznawać akordy muzyczne. Ich pomysł zapoczątkował całą serię badań innych naukowców zajmujących się problemem ACR (ang. \emph{Audio Chord Recognition}), stanowiącym jedno z podstawowych zadań z dziedziny MIR (ang. \emph{Music Information Retrieval}). Zaproponowane przez nich rozwiązanie jest stosunkowo proste, dzisiaj natomiast już zdecydowanie wymagające usprawnień. Wynika to z faktu, że faktyczny rozwój uczenia głębokiego (ang. \emph{deep learning}) rozpoczął się właśnie w roku 2012 i od tego czasu powstało całe mnóstwo metod pozwalających osiągnąć dokładniejsze wyniki mniejszym kosztem obliczeniowym.

% opis prac Korzeniowskiego
Jedną z najobszerniejszych serii prac dotyczących rozpoznawania akordów muzycznych z wykorzystaniem sieci neuronowych wykonali Korzeniowski i Widmer \cite{korzeniowski_feature_2016,korzeniowski_fully_2016,korzeniowski_futility_2017,korzeniowski_improved_2018,korzeniowski_automatic_2018,korzeniowski_large-scale_2018}. W ciągu kilku lat przeprowadzili szereg badań i wydali sześć artykułów związanych z tym tematem. Artykuły te wynikają kolejno jedne z drugich i nawzajem się uzupełniają. W pierwszym z nich \cite{korzeniowski_feature_2016} autorzy proponują użycie perceptronu wielowarstwowego do ekstrakcji cech dźwięku (ang. \emph{chroma feature}) w miejsce wcześniej stosowanych deterministycznych, znacznie mniej skomplikowanych algorytmów ekstrakcji cech. Tak powstała reprezentacja jest ich zdaniem znacznie lepsza i może zostać wykorzystana do dalszej klasyfikacji akordów, lub w zupełnie innym celu. Dodatkowo pomysł ten jest motywowany założeniem, że to właśnie ekstrakcja cech z surowych danych gra kluczową rolę w jakości klasyfikacji akordów --- jest znacznie ważniejsza od późniejszego rozpoznawania konkretnego akordu i detekcji całej sekwencji. W kolejnej pracy \cite{korzeniowski_fully_2016} autorzy wykorzystują tym razem splotową sieć neuronową w połączeniu z CRF (ang. \emph{Conditional Random Fields}) aby stworzyć kompletny algorytm rozpoznający akordy. Daje on bardzo konkurencyjne wyniki, które do dziś praktycznie nie zostały znacząco poprawione. Rozwiązania te stanowią punkt odniesienia dla wielu kolejnych prac innych badaczy \cite{ohanlon_fifthnet_2021, park_bi-directional_2019}.

W dalszym toku badań autorzy skupiają się na możliwości usprawnienia wcześniejszych rozwiązań poprzez wykorzystanie modeli językowych i powiązania między zadaniem detekcji sekwencji akordów a zadaniem detekcji i zrozumienia sekwencji słów w języku naturalnym. Najpierw udowadniają, że złożone modele językowe (sieci rekurencyjne) nie sprawdzą się dobrze, jeśli będą stosowane na poziomie pojedynczych ramek czasowych, a nie pojedynczych wystąpień akordów \cite{korzeniowski_futility_2017}. W takiej sytuacji lepiej sprawdzają się znacznie prostsze modele jak HMM (ang. \emph{Hidden Markov Model}), które w praktyce jedynie ,,wygładzają'' sekwencję akordów. W kolejnych badaniach \cite{korzeniowski_large-scale_2018} wykazują, że modele stosowane do przetwarzania języka naturalnego (ang. \emph{Natural Language Processing}) mają duży potencjał i potrafią skutecznie modelować zależności w sekwencjach akordów muzycznych (np. przewidywać cykle), co pozwala usprawnić ogólną jakość klasyfikacji akordów. Autorzy opisują w końcu model probabilistyczny, pozwalający połączyć model akustyczny (rozpoznający akordy w danej chwili czasu) z modelem językowym (dekodującym całą sekwencję pojedynczych akordów), implementują go i przeprowadzają szereg eksperymentów pozwalających potwierdzić, że tak zastosowane złożone modele językowe usprawniają jakość klasyfikacji akordów \cite{korzeniowski_improved_2018}.

O ile opisane powyżej prace stanowią bardzo istotny wkład w dziedzinę badań nad zadaniem ACR i porządkują wiele aspektów tego zagadnienia (jak kwestia wykorzystania modeli językowych), to w obecnej chwili są już raczej przestarzałe. Spostrzeżenia i wnioski zawarte w tych pracach pozostają w większości aktualne, jednakże nie ma sensu stosowanie wykorzystywanych wtedy architektur sieci neuronowych, ze względu na możliwość wykorzystania nowszych modeli, dokładniejszych i wydajniejszych obliczeniowo.

% structured training (o niezbalansowaniu)
Równolegle do prac Korzeniowskiego i Widmera, jak również już po ich ostatnich publikacjach związanych z tym tematem, powstawało wiele innych prac dotyczących zadania rozpoznawania akordów muzycznych za pomocą sieci neuronowych. Wśród nich warto wspomnieć \cite{mcfee_structured_2017}, gdzie autorzy próbują rozwiązać problem niezbalansowanych zbiorów danych (w praktyce niektóre akordy występują znacznie rzadziej niż inne, a do tego są bardzo podobne do tych występujących częściej) poprzez ,,ustrukturyzowanie'' treningu. Rozwiązanie to w uproszczeniu polega na dołożeniu dodatkowych funkcji kosztu dla sieci neuronowej (dodatkowego nadzorowania), wymuszających ekstrakcję pewnych konkretnych (wspólnych między akordami) cech. Wszystko to pozwala osiągnąć trochę lepsze wyniki, jednakże wadą jest skomplikowanie rozwiązania. Ponadto sama idea i kierunek badań mogą zostać poddane w wątpliwość, ze względu na dużą ingerencję w sposób działania sieci neuronowej i wymuszanie pewnych zachowań. Alternatywą jest dążenie w kierunku generalizacji i faktycznego polegania na danych uczących, a nie na wiedzy dziedzinowej (tzw. podejście \emph{data driven}). Przykładem takiego kierunku badań są właśnie metody nienadzorowanego uczenia sieci neuronowych.

% użycie transformerów
W przetwarzaniu języka naturalnego już od kilku lat standardem jest architektura zwana Transformerem \cite{vaswani_attention_2017}, która pozwala zrównoleglić obliczenia wcześniej przeprowadzane sekwencyjne i w ogóle pozwala osiągnąć wyniki lepsze, niż stosowane wcześniej do tego zadania sieci rekurencyjne. Niedawno architektura ta weszła również do użytku w przetwarzaniu obrazów \cite{dosovitskiy_image_2021} i obecnie modele tego typu osiągają wyniki lepsze niż klasyczne splotowe sieci neuronowe. Jeszcze przed wprowadzeniem Transformerów do przetwarzania obrazów wykonane zostały co najmniej dwie próby wykorzystania tej architektury do zadania rozpoznawania akordów. Pierwsza z nich \cite{chen_harmony_2019} jest trudna do oceny ze względu na małą liczbę przeprowadzonych eksperymentów oraz brak realnego odniesienia do rozwiązań innych autorów. Natomiast w \cite{park_bi-directional_2019} pokazane zostało, że Transformer może dać wyniki bardzo zbliżone do osiąganych w \cite{korzeniowski_fully_2016}. Chociaż transformer nie pozwolił osiągnąć lepszych wyników, to należy pamiętać, że była to zaledwie pierwsza próba wykorzystania tej architektury, która w najprostszej postaci sprawdziła się zupełnie zadowalająco. Można dodać, że po wprowadzeniu transformerów do przetwarzania obrazów, zostały one również w podobny sposób wykorzystane do klasyfikacji dźwięków \cite{gong_ast_2021}.

% fifthnet
Warto jeszcze wspomnieć o dwóch bardzo istotnych pracach, jakimi są \cite{hanlon_fifthnet_2020} oraz \cite{ohanlon_fifthnet_2021}. Pomysł autorów polega tam na radykalnym zmniejszeniu liczby parametrów sieci, na rzecz specjalistycznej architektury, odpowiedniej do problemu rozpoznawania akordów. Udało się wykazać tam, że wielokrotnie mniejszy model jest w stanie osiągnąć tak samo dobre wyniki, jak \cite{korzeniowski_fully_2016}, przy odpowiednim formacie danych wejściowych i odpowiednim układzie warstw sieci neuronowej. Prace te są o tyle ważne, że pokazują, iż problem rozpoznawania akordów nie jest bardzo złożony, zwłaszcza w porównaniu z takimi zadaniami, jak klasyfikacja obrazów czy rozpoznawanie mowy.

\subsection{Nienadzorowane uczenie sieci neuronowych}

% chaotycznie o uczeniu nienadzorowanym
Choć próby realizacji uczenia sieci neuronowych bez oznaczeń były stosowane już wcześniej \cite{noroozi_unsupervised_2017}, to stały się tak naprawdę przydatne w momencie, kiedy wykorzystano je w przetwarzaniu języka naturalnego \cite{devlin_bert_2019}. W tym obszarze stosowanie treningów nienadzorowanych, a dokładnie samonadzorowanych, stało się właściwie standardem. Jeżeli chodzi o obszar przetwarzania obrazów, to tutaj sytuacja ma się nieco inaczej. Wcześniejsze próby, takie jak \cite{pathak_context_2016}, czy \cite{noroozi_unsupervised_2017} nie przyniosły szczególnych rezultatów. Wciąż bardziej opłacało się trenować na dużych zbiorach do klasyfikacji i ewentualnie dotrenowywać na zadaniu docelowym. Sytuacja ta wynika również ze znacznie większej niż w przypadku analizy tekstu naturalnego, dostępności danych oznaczonych. Jednakże z czasem, coraz więcej nowych pomysłów zaczęło się pojawiać i powstały między innymi takie serie prac jak \cite{chen_simple_2020, grill_bootstrap_2020, caron_unsupervised_2021, caron_emerging_2021}, które dotyczą stricte samonadzorowanego uczenia na obrazach poprzez sprytne wykorzystanie augmentacji. Metody te nie rozpowszechniły się tak bardzo, jak w innych obszarach, być może również ze względu na swoją złożoność. Nie wydają się one również tak dobrze pasować do zadania rozpoznawania akordów, które jednak znacznie bardziej zbliżone jest do przetwarzania tekstu, a najbardziej do innych form analizy dźwięku, np. rozpoznawania mowy. To właśnie w tym ostatnim obszarze, podobnie jak w przetwarzaniu języka naturalnego, bardzo powszechnym stało się korzystanie z metod samonadzorowanych, takich jak \cite{baevski_wav2vec_2020}. Innymi przykładami stosowania metod samonadzorowanych w analizie dźwięku są dwie bardzo podobne prace \cite{baade_mae-ast_2022} i \cite{he_masked_2021}. Obie bazują na kluczowym dla niniejszej pracy pomyśle o nazwie MAE \cite{he_masked_2021}, który pochodzi akurat z dziedziny przetwarzania obrazów, ale tak naprawdę jest jedynie udaną adaptacją, pochodzącego z przetwarzania języka naturalnego, słynnego \cite{devlin_bert_2019}.

% bardziej o semi-supervised
Po części alternatywnym pomysłem do treningów samonadzorowanych są treningi półnadzorowane. Warto wspomnieć o dwóch bardzo udanych pracach, jakimi są związane z przetwarzaniem obrazów \cite{xie_self-training_2020} oraz \cite{pham_meta_2021}. W przeciwieństwie do metod samonadzorowanych algorytmy te okazały się bardzo praktyczne i zostały wykorzystane przy osiąganiu najwyższych wyników w szeroko rozumianej klasyfikacji obrazów. Co ciekawe, jeżeli chodzi o metody nienadzorowane w rozpoznawaniu akordów, to również należą one bardziej do grupy półnadzorowanych. Powstały zwłaszcza dwie takie prace. Pierwszą jest \cite{wu_semi-supervised_2020}, w której autorzy wykorzystują autoenkoder wariacyjny. Jeżeli chodzi o uzyskaną poprawę, to zaznaczają oni, że wykorzystanie zaproponowanego przez nich algorytmu poprawia jakość klasyfikacji w stosunku do prostego uczenia nadzorowanego na tych samych danych. Niestety nie porównują się do innych algorytmów. Pokazują natomiast wpływ proporcji danych oznaczonych i nieoznaczonych na otrzymaną dokładność. Drugim algorytmem jest \cite{bortolozzo_improving_2021}, w którym autorzy zaadaptowali wspomniany wcześniej \cite{xie_self-training_2020} do zadania rozpoznawania akordów. Jeżeli chodzi o uzyskaną poprawność, to wyniki prezentowane w tej pracy są całkiem spektakularne --- wielokrotnie poprawia się dokładność klasyfikacji rzadko występujących akordów.



\section{Układ pracy}

Tematem pracy jest: ,,Nienadzorowane uczenie sieci neuronowych w zadaniu rozpoznawania akordów muzycznych'', za główny zaś cel przyjęto zaproponowanie i przetestowanie algorytmu nienadzorowanego uczenia sieci neuronowych w zadaniu rozpoznawania akordów. Rozdział \ref{chapter:introduction} zawiera wstęp i cele pracy oraz przegląd literatury. Rozdział \ref{chapter:music_theory} wprowadza niezbędny zestaw pojęć z obszaru cyfrowego przetwarzania sygnałów i teorii muzyki oraz wyjaśnia, czym właściwie są akordy. W rozdziale \ref{chapter:dataset} szczegółowo przedstawiony został proces gromadzenia i wstępnego przetwarzania zbiorów danych. Następnie w rozdziale \ref{chapter:methodology} opisany został dokładnie zaproponowany algorytm oraz wprowadzone zostały niezbędne pojęcia z dziedziny uczenia maszynowego i sieci neuronowych. Rozdział \ref{chapter:experiments} prezentuje szczegółowy opis przeprowadzonych eksperymentów oraz wyniki i ich dyskusję. W podsumowaniu pracy przedstawiono najważniejsze obserwacje i wnioski z przeprowadzonych eksperymentów, z których wynika, że metody nienadzorowane są przydatne i mają potencjał, aby być stosowane w zadaniu rozpoznawania akordów i innych zadaniach z dziedziny MIR. Wnioski te stanowią najważniejszy rezultat niniejszej pracy.
